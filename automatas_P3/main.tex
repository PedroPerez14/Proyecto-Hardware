\documentclass[landscape,a4paper]{article}
\usepackage[margin=1.5cm]{geometry}
\usepackage{underscore}
\usepackage[utf8]{inputenc}

\usepackage{float}
\usepackage{booktabs}

% Comando para poner la etiqueta de un arco del grafo de forma horizontal
\newcommand{\ltag}[2]{\ttfamily\footnotesize \begin{tabular}{l}#1\end{tabular}\rmfamily\huge/\normalsize\ttfamily\footnotesize\begin{tabular}{l}#2\end{tabular}}
% Comando para poner la etiqueta de un arco del grafo de forma vertical
\newcommand{\ptag}[2]{\ttfamily\footnotesize \begin{tabular}{l}#1\end{tabular}\\\rmfamily\huge/\normalsize\ttfamily\footnotesize\begin{tabular}{l}#2\end{tabular}}
% Espacio horizontal para simular tabulado dentro de entorno tabular
\newcommand{\tab}{\hspace{0.2cm}}


\usepackage[spanish]{babel}
\usepackage{tikz}
\usetikzlibrary{automata, positioning, arrows, shapes.geometric}
\tikzset{
	->, % hace que los arcos sean dirigidos
	>=stealth, % las cabezas de las flechas son en negrita
	node distance=3cm, % Especifica la mínima distancia entre dos nodos, se puede cambiar si es necesario.
	every state/.style={thick, fill=gray!10}, % establece las propiedades para cada nodo estado. En tickz cada objeto que se pone es un nodo. Se pueden poner arcos entre esos nodos.
	initial text=$ $, % establece el texto que aparece en la flecha inicial
	elliptic state/.style={draw,ellipse},
}

\definecolor{error}{rgb}{1,0.4,0.4}

\definecolor{bien}{rgb}{0.8, 1, 0.5}

\begin{document}
{\Large \bf \sffamily Práctica 3 Proyecto Hardware - Máquinas de estados}

{\sffamily \large Fernando Peña Bes (NIA: 756012) y Pedro José Pérez García (NIA: 756642)}

\emph{\sffamily \large  Curso 2019-20, Universidad de Zaragoza}

\vspace{2cm}
\begin{figure}[h]
\centering
\begin{tikzpicture}[thick,scale=1, every node/.style={scale=1}]
	\node[elliptic state, initial, align=center] (q0) {\bf Reglas};
	\node[elliptic state, above of=q0, xshift=7cm, yshift=5cm, align=center] (q1) {\bf Fin};
	\node[elliptic state, right of=q0, xshift=10cm, align=center] (q2) {\bf Jugando};
	
	
	\draw

	% Estado Reglas
	(q0) edge[loop, in=100, out=130, looseness=10, above, align=center] node {\ptag{ev_bot_iz == 1}{ev_bot_iz = 0}} (q0)
	(q0) edge[loop, in=220, out=240, looseness=20, below, align=center] node {\ptag{ev_bot_der == 1}{ev_bot_der = 0}} (q0)
	(q0) edge[loop, in=300, out=320, looseness=10, below right, align=center] node {\ptag{ev_tp == 1 \&\& \\ haciendo_DMA == 1}{ev_tp = 0}} (q0)

	(q0) edge[below, align=center] node {\ptag{ev_tp == 1 \&\& haciendo_DMA == 0}{ev_tp = 0, \\ haciendo_DMA = 1, \\ pintar_jugando()}} (q2)


	% Estado Jugando
	(q2) edge[loop, out=210, in=230, looseness=30, below, align=left] node {\ptag{haciendo_DMA == 0 \&\& \\ ev_timer >\ 0}{\\ev_timer--, \\ tiempo_total++, \\ haciendo_DMA = 1, \\ pintar_profiling(tiempo_total, ...) }} (q2)
	

	(q2) edge[loop, out=330, in=340, looseness=70, below right, align=center] node {\ptag{ev_bot_der == 1 \&\& \\ haciendo_DMA == 1}{ev_bot_der = 0}} (q2)
	(q2) edge[loop, out=300, in=320, looseness=50, below, align=center] node {\ptag{ev_bot_izq == 1 \&\& \\ haciendo_DMA == 1}{ev_bot_izq = 0}} (q2)
	

	(q2) edge[loop, out=350, in=360, looseness=40, right, align=center] node {\ptag{ev_tp == 1 \&\& \\ haciendo_DMA == 1}{ev_tp = 0}} (q2)


	(q2) edge[loop, out=20, in=30, looseness=70, right, align=left] node {\ptag{ev_bot_der == 1 \&\& haciendo_DMA == 0}{\\ev_bot_der = 0, \\ haciendo_DMA = 1, \\ calcular_mov_gris(), \\ mover_gris(mi_fila, mi_columna, aux, aux2)}} (q2)
	(q2) edge[loop, out=70, in=90, looseness=80, above right, align=left] node {\ptag{ev_bot_izq == 1 \&\& haciendo_DMA == 0}{\\ev_bot_izq = 0, \\ haciendo_DMA = 1, \\ calcular_mov_gris(), \\ mover_gris(mi_fila, mi_columna, aux, aux2)}} (q2)



	(q2) edge[loop, out=140, in=160, looseness=20, left, align=center] node {\ptag{ev_tp == 1 \&\& \\ haciendo_DMA == 0}{haciendo_DMA = 1, \\ procesar_jugada()}} (q2)


	% A estado fin

	(q2) edge[above, align=center, sloped] node {\ptag{haciendo_DMA == 0 \&\& final != no_fin}{pintar_fin(), \\ haciendo_DMA = 1}} (q1)



	% Estado Fin
	(q1) edge[align=center, above, sloped] node {\ptag{haciendo_DMA == 0 \&\& ev_tp == 1}{ev_tp = 0, LCD_Clr(), \\ inicializar_jugada_botones()}} (q0)

	
	%(q3) edge[loop, in=30, out=60, looseness=8, right] node{\ltag{latido 0 de C\\ E = []} {nuevaVista \\ V = \{_,_,_\} \\ T = \{n+1,P,C\} \\ send T}} (q3)

	% Q4
	% Q4 a Q5
	
	
	
	
	% Recibir de nodos que no son ni P ni C
	%(q4) edge[loop, out=40, in=65, looseness=20, sloped, above, align=center] node{\ptag{latido != (P|C) \\ latido not 0 de P o C}{manda T \\ \emph{if} not in E, \\ \tab\tab \emph{do:} E ++ [nodo]}} (q4)


			   ;
\end{tikzpicture}
\caption{Autómata de \texttt{jugada_por_botones.c}}
\end{figure}

\pagebreak

\begin{figure}[h]
	\centering
	\begin{tikzpicture}[thick,scale=1, every node/.style={scale=1}]
		\node[elliptic state, initial, align=center] (q1) {\bf  Inicio};
		\node[elliptic state, right of=q1, xshift=10cm, align=center] (q2) {\bf Contando \tt trp};
		\node[elliptic state, below of=q1, yshift=-7cm, align=center] (q4) {\bf Contando \tt trd};
		\node[elliptic state, below of=q2, yshift=-7cm, align=center] (q3) {\bf Encuestando cada 30\,ms};
		
		\draw 	
		
		
				(q1) edge[below, align=center] node{\ptag{atendiendoPulsacion == 1}{cuenta_ticks = 0}} (q2)

				(q2) edge[above, align=center, sloped] node{\ptag{cuenta_ticks == trp_en_ticks_timer0}{cuenta_ticks = 0}} (q3)




				(q3) edge[loop, right, out=20, in=10, looseness=15, align=left] node{        \ptag{cuenta_ticks == TICKS_PARA_30_MS \&\& \\ button_estado() == que_button \&\& \\ cuenta_autoincremento == int_timer0_enable_autoincr \&\& \\ que_button == button_iz}{cuenta_autoincremento = 0, \\ hay_autoincremento = 1, \\ jugada_ev_izq()}} (q3)
				(q3) edge[loop, below right, out=345, in=350, looseness=22, align=left] node{\ptag{cuenta_ticks == TICKS_PARA_30_MS \&\& \\ button_estado() == que_button \&\& \\ cuenta_autoincremento == int_timer0_enable_autoincr \&\& \\ que_button == button_dr}{cuenta_autoincremento = 0, \\ hay_autoincremento = 1, \\ jugada_ev_der()}} (q3)
				(q3) edge[loop, above, out=140, in=160, looseness=30, align=left] node{      \ptag{cuenta_ticks == TICKS_PARA_30_MS \&\& \\ button_estado() == que_button \&\& \\ hay_autoincremento \&\& \\ cuenta_autoincremento == int_timer0_enable_autoincr \&\& \\ que_button == button_iz}{cuenta_autoincremento = 0, \\ jugada_ev_izq()}} (q3)
				(q3) edge[loop, out=260, in=290, looseness=30, below, align=left] node{      \ptag{cuenta_ticks == TICKS_PARA_30_MS \&\& \\ button_estado() == que_button \&\& \\ hay_autoincremento \&\& \\ cuenta_autoincremento == int_timer0_enable_autoincr \&\& \\ que_button == button_dr}{cuenta_autoincremento = 0, \\ jugada_ev_der()}} (q3)






				(q3) edge[below, align=center] node{\ptag{cuenta_ticks == TICKS_PARA_30_MS \&\& \\ button_estado() != que_button}{cuenta_ticks = 0, \\ cuenta_autoincremento = 0}} (q4)

				(q4) edge[above, align=center, sloped] node{\ptag{cuenta_ticks == trd_en_ticks_timer0}{button_resetear(), \\ atendiendoPulsacion = 0,\\  cuenta_ticks = 0}} (q1)


				;
	\end{tikzpicture}
	\caption{Autómata de \texttt{botones_antirebotes.c}}
	\end{figure}

	\pagebreak

	\begin{figure}[h]
		\centering
		\begin{tikzpicture}[thick,scale=1, every node/.style={scale=1}]
			\node[elliptic state, initial, align=center] (q1) {\bf  Inicio};
			\node[elliptic state, right of=q1, xshift=10cm, align=center] (q2) {\bf Esperar \tt trd};
			
			\draw 	
			
			
					(q1) edge[above, bend left, align=center] node{\ptag{atendiendoPulsacion == 1}{-}} (q2)
	
					(q2) edge[below, bend left, align=center] node{\ptag{cuenta_ticks_tsp == t_tsp_espera_ticks_timer0}{tsp_resetear(), \\ tsp_antirebotes_inicializar()}} (q1)
	
					;
		\end{tikzpicture}
		\caption{Autómata de \texttt{tsp_antirebotes.c}}
		\end{figure}

		\pagebreak

		\begin{figure}[h]
			\centering
			\begin{tikzpicture}[thick,scale=1, every node/.style={scale=1}]
				\node[elliptic state, initial, align=center] (q1) {\bf  Inicio};
				\node[elliptic state, right of=q1, xshift=10cm, align=center] (q2) {\bf Esperar \tt trd};
				
				\draw 	
				
				
						(q1) edge[above, bend left, align=center] node{\ptag{atendiendoPulsacion == 1}{-}} (q2)
		
						(q2) edge[below, bend left, align=center] node{\ptag{cuenta_ticks_tec == t_tec_espera_ticks_timer0}{tec_resetear(), \\ tec_antirebores_inicializar()}} (q1)
		
						;
			\end{tikzpicture}
			\caption{Autómata de \texttt{teclado_antirebotes.c}}
			\end{figure}

				   
		       	
	
\end{document}
